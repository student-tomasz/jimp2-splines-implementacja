\documentclass[10pt,a4paper]{article}
\usepackage[a4paper]{geometry}

\usepackage{polski}
\usepackage{xltxtra}

\usepackage{fancyvrb}
\usepackage{relsize}
\usepackage{alltt}
\usepackage{amsfonts}
\usepackage[pdfborder={0 0 0}]{hyperref}

%% tweak fonts
\defaultfontfeatures{Mapping=tex-text}
\setromanfont{Charis SIL}
%\setsansfont[Scale=MatchLowercase]{Gill Sans}
\setmonofont[Scale=MatchLowercase]{Menlo}
\linespread{1.25}

%% define custom commands and environments
\DefineVerbatimEnvironment%
  {SmallVerbatim}%
  {Verbatim}{fontsize=\relsize{-0.5},numbers=left,numbersep=-10pt,frame=lines,tabsize=4}
\newcommand{\p}[1]{\texttt{#1}}
\newcommand{\flag}[1]{\textbf{\p{#1}}}

\begin{document}

%%fakesection{Tytuł}
\title{ 
  Interpolacja funkcjami sklejanymi\\
  {\normalsize Specyfikacja implementacyjna projektu nr 1}\\\vspace{-12pt}
  {\normalsize z przedmiotu \emph{Języki i metody programowania 2}}
}
\author{
  Tomasz Cudziło\\
  {\small EE PW, 211552}
}
\date{\today}
\maketitle

\section*{Zadanie}
\label{sec:zadanie}

Napisać program wyznaczający współczynniki funkcji sklejanych trzeciego stopnia
aproksymujących zadany ciąg danych pomiarowych.

\vspace{24pt}

\section{Moduły}
Program jest podzielony na cztery moduły współpracujące ze sobą: \p{CLI},
\p{IO}, \p{Interpolation}, \p{Math}.

\subsection{Moduł \p{CLI}}
Składa się z~plików \p{options.h}, \p{cli.h} i~\p{cli.c}. Udostępnia strukturę
\p{options} i~funkcję \p{cli\_parse()}.
\begin{SmallVerbatim}
    /* file: options.h */
    struct {
      char *in_filename;
      char *out_filename;
      int force;
      int gnuplot;
      int quiet;
    } *options = NULL;
\end{SmallVerbatim}
Ma prawo istnieć tylko jedna instancja struktury \p{options} i~jest ona
globalna. Obiektowo \p{options} byłyby klasą zaimplementowaną jako singleton.
\begin{SmallVerbatim}
    /* file: cli.h */
    void cli_parse(char **argv, int argc);
\end{SmallVerbatim}
Instancja \p{options} jest inicjalizowana w~funkcji \p{cli\_parse()} i~nie jest
zmieniana w~pozostałych częściach programu.

Pola \p{force}, \p{gnuplot} i~\p{quiet} ustawiane są na \p{1} lub \p{0}
zależnie od wartości domyślnych i~flag podanych przez użytkownika
w~argumentach~linii komend.

Wskaźniki \p{in\_filename} i~\p{out\_filename} wskazują na string ze~względną
ściężką odpowiednio do pliku wejściowego i~wyjściowego. Jeśli pliki nie
zostały podane, wskaźniki są \p{NULL}. Dla \p{in\_filename == NULL} oznacza to
wczytywanie danych z~wejścia standardowego. Specyfikacja funkcjonalna opisuje
zależności między nazwami plików tworzonych przez program --- co najwyżej jeden
z~tych wskaźników może być \p{NULL}.

\subsection{Moduł \p{IO}}

Moduł \p{IO} dostarcza funkcji do obsługi plików związanych z~programem.
Zawiera pliki \p{io.h} i~\p{io.c}. Formaty obsługiwanych plików i~przypadki ich
tworzenia są opisane w~specyfikacji funkcjonalnej. Wykorzystuje struktury
zdefiniowane w~module \p{Math} (par.~\ref{sec:modul_math}).
\begin{SmallVerbatim}
    /* file: io.h */
    point_t *io_read();
    int io_write(point_t **nodes, polynomial_t **splines);
    void io_log(const char *msg);
    void io_error(const char *msg);
\end{SmallVerbatim}
\begin{description}
  \item[\p{io\_read()}] \hfill \\
    Wczytuje i~parsuje dane z~pliku lub z~\p{stdin}. Zwraca wskaźnik na
    początek dynamicznie zaalokowanej tablicy wskaźników na punkty lub \p{NULL}
    w~przypadku niepowodzenia. Sortuje rosnąco punkty według ich współrzędnej
    $x$.
  \item[\p{io\_write()}] \hfill \\
    Otrzymuje tablicę wskaźników na punkty wyznaczające przedziały oraz tablicę
    wskaźników na wielomiany funkcji sklejanej. Kolejność wielomianów odpowiada
    kolejności przedziałów w~tablicy węzłów. Otrzymane dane wypisuje do pliku
    wyjściowego i~do pliku dla \p{gnuplot}. Zwraca \p{0} w~przypadku
    niepowodzenia zapisu do pliku.
  \item[\p{io\_log()}] \hfill \\
    Wypisuje wiadomość informacyjną do pliku z~logiem.
  \item[\p{io\_error()}] \hfill \\
    Wypisuje wiadomości o~błędach. Działa identycznie jak \p{io\_log()}.
\end{description}

\subsection{Moduł \p{Math}}
\label{sec:modul_math}

Jest modułem pomocniczym z~podstawowymi narzędziami matematycznymi, których nie
ma w~bibliotece standardowej.

\subsubsection{Pliki \p{point.h} i~\p{point.c}}
Daje prostą reprezentację punktu na $\mathbb{R}^{2}$ oraz konstruktor i~destruktor.
\begin{SmallVerbatim}
    /* file: point.h */
    typedef struct point_s {
      double x;
      double y;
    } point_t;
    
    point_t *point_new(double x, double y);
    void point_free(point_t *p);
\end{SmallVerbatim}

\subsubsection{Pliki \p{polynomial.h} i~\p{polynomial.c}}
Definiuje strukturę odpowiadającą funkcji wielomianowej.
\begin{SmallVerbatim}
    /* file: polynomial.h */
    typedef struct polynomial_s {
      double *coeffs;
      int degree;
    } polynomial_t;
    
    polynomial_t *polynomial_new(double *coeffs, int degree);
    void polynomial_free(polynomial_t *p);
    polynomial_t *polynomial_derivate(polynomial_t *p);
\end{SmallVerbatim}
\p{coeffs} jest dynamicznie alokowaną tablicą liczb rzeczywistych
o~$\p{degree}+1$ elementach. Każdy element \p{coeffs} odpowiada współczynnikowi
zmiennej w~potędze równej indeksowi tego elementu. To znaczy dla wielomianu
\ref{eqn:poly} struktura posiada tablicę \ref{eqn:array}:
\begin{eqnarray}
  4.4 \cdot x^{3} +~1.2 \cdot x^2 - 3.1 \cdot x + 0.7 \label{eqn:poly}\\
  {\tt\left[ 0.7,\quad -3.1,\quad 1.2, \quad 4.4 \right]} \label{eqn:array}
\end{eqnarray}
Funkcja \p{polynomial\_derivate()} tworzy nowy wielomian będący funkcją
pochodną wielomianu przekazanego w~argumencie.

\subsubsection{Pliki \p{matrix.h} i~\p{matrix.c}}
Zawiera reprezentację macierzy i~metodę rozwiązywania układów równań metodą
eliminacji Gaussa. Jeśli te proste rozwiązania okażą się za wolne, zostaną
zoptymalizowane. \emph{Premature optimization is the root of all evil.}
\begin{SmallVerbatim}
    /* file: matrix.h */
    typedef struct matrix_s {
      double *t;
      int r;
      int c;
    } matrix_t;
    
    matrix_t *matrix_new(double *t, int r, int c);
    void matrix_free(matrix_t *m);
    matrix_t *matrix_gauss(matrix_t *A, matrix_t *b);
\end{SmallVerbatim}

\subsection{Moduł \p{Spline}}

Moduł ma dwa pliki \p{spline.h} i~\p{spline.c}. Zawiera algorytm interpolacji.
\begin{SmallVerbatim}
    /* file: spline.h */
    polynomial_t **spline_interpolate(point_t **nodes);
\end{SmallVerbatim}
Funkcja \p{spline\_interpolate()} otrzymuje tablicę wskaźników na punkty podane
przez użytkownika. Zwraca tablicę wskaźników na wielomiany tworzące szukaną
funkcję sklejaną. Funkcja zakłada, że punkty w~\p{nodes} są posortowane
rosnąco i~zwraca wielomiany w~kolejności odpowiadającej przedziałom
z~\p{nodes}.

Algorytm interpolacji sprowadza się do rozwiązania układu równań z~założenia
równości wielomianów i~ich pochodnych na krańcach przedziałów zdefiniowanych
przez punkty.
\begin{SmallVerbatim}
    /* file: pseudokod spline.c */
    spline_interpolate(punkty)
    {
      n = liczba punktow;
      A = macierz kwadratowa o wymiarze n-1;
      b = macierz o n-1 wierszach i 1 kolumnie;
      
      uzupelnij macierze A i b z założeń:
      * równości wielomianów w punktach granicznych przedziałów;
      * równości pochodnych 1- i 2-stopnia tych wielomianów w tych samych punktach;
      
      k = matrix_gauss(A, b);
      wielomiany = wyciągnij z k współczynniki poszczególnych wielomianów;
      
      return wielomiany;
    }
\end{SmallVerbatim}

\section{Integracja}

Program nie jest złożony. Cała logika mieści się w~\p{main.c}:
\begin{SmallVerbatim}
    /* file: pseudokod main.c */
    main()
    {
      globalne_opcje = cli_parse();

      punkty = io_read();
      wielomiany = spline_interpolate();
      io_write(punkty, wielomiany);
      
      return sukces;
    }
\end{SmallVerbatim}

\end{document}
