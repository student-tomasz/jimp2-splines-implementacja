\documentclass[12pt,a4paper]{article}
\usepackage[a4paper,top=3cm,bottom=3cm]{geometry}

\usepackage[T1]{fontenc}
\usepackage{polski}

\usepackage{xltxtra}
\defaultfontfeatures{Mapping=tex-text}
\setromanfont{Gentium}
%\setsansfont[Scale=MatchLowercase]{Gill Sans}
%\setmonofont[Scale=MatchLowercase]{Monaco}
\linespread{1.25}

\usepackage{fancyvrb}
\usepackage{relsize}
\usepackage{alltt}

\usepackage[pdfborder={0 0 0}]{hyperref}

\DefineVerbatimEnvironment%
  {SmallVerbatim}%
  {Verbatim}{fontsize=\relsize{-1},numbers=left,numbersep=-10pt,frame=lines,tabsize=4}

\newcommand{\prog}[1]{\texttt{#1}}
\newcommand{\flag}[1]{\textbf{\prog{#1}}}

\begin{document}

%%fakesection{Tytuł}
\title{ 
  Interpolacja funkcjami sklejanymi\\
  {\normalsize Specyfikacja implementacyjna projektu nr 1}\\\vspace{-12pt}
  {\normalsize z przedmiotu \emph{Języki i metody programowania 2}}
}
\author{
  Tomasz Cudziło\\
  {\small EE PW, 211552}
}
\date{\today}
\maketitle

\section*{Zadanie}
\label{sec:zadanie}

Napisać program wyznaczający współczynniki funkcji sklejanych trzeciego stopnia
aproksymujących zadany ciąg danych pomiarowych.

\vspace{24pt}

\end{document}
